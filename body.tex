% For formatting purposes only 
\setcounter{chapter}{8}
\setcounter{section}{3}
\setcounter{subsection}{0}
%\listofalgorithms - think about it ...
%--------------
%- COPY START -
%--------------
\paragraph{Frazzoli Movement Automaton:} The following paragraph strongly follows Frazzoli work \cite{frazzoli2001robust} (sec 3.1-3.5).


The approach was proposed to reduce \emph{computational complexity problem} of \emph{motion planning}. The quantization of the system dynamics is done through restriction of feasible nominal system trajectories to the \emph{family of time parametrized curves}. These can be obtained by interconnection of trajectory primitives.

\emph{Trajectory primitives} are repeatable portions of trajectory (def. \cite{frazzoli2001robust}.3.1). The trajectory primitives are interconnected by \emph{transitions} to create maneuvers (movements) (def. \cite{frazzoli2001robust}.3.3). 

By combining movements as a set of trim trajectories the trajectory can be represented as set of discrete time bounded commands. This is summarized in definition (def. \cite{frazzoli2001robust}.3.4) based on \emph{hybrid automaton definition} (sec. \ref{s:HybridAutomaton}).
\begin{definition}[Maneuver Automaton] A maneuver automaton over a mechanical control system $S$, with symmetry group $H$ is described by the following objects:

\begin{enumerate}
    \item A finite set of indices $Q = Q_T \cup Q_M \in \N$, where the subscript T relates to trim trajectories, and the subscript M relates to maneuvers;
    
    \item A finite set of trim trajectory parameters $\left(\bar{g},\bar{\xi},\bar{u}\right)_q$; with $q\in Q_T$.
    
    \item A finite set of maneuver parameters, and state and control trajectories $\left(T,u,\phi\right)_q$, with $q\in Q_M$.
    
    \item The maps $Previous: Q_M\to Q_T $, and, $Next: Q_M \to Q_T$ such that $Previous(q)$ and $Next(q)$ give, respectively, the index of the trim trajectories from which the maneuver $q$ starts and ends.
    
    \item A discrete state $q \in Q$.
    
    \item A continuous state, denoting the position on the symmetry group, $h \in H$.
    
    \item A clock state $\theta \in \R$, which evolves according to $\dot{\theta}=1$, and which is reset after each switch on $q$.
\end{enumerate}
\begin{note}
    It is apparent tat decisions can be made about the future evolution of the system only when the system is executing a trim trajectory (that is, the discrete state is in one of the nodes in the graph). While executing a maneuver the system is committed to it, and must keep executing the maneuver until its completion. As a consequence, for motion planning and control design purposes, one can concentrate the study of the evolution of the system on and between nodes.
\end{note}
\end{definition}