% For formatting purposes only 
\setcounter{chapter}{6}
\setcounter{section}{7}
\setcounter{subsection}{3}
%\listofalgorithms - think about it ...
%--------------
%- COPY START -
%--------------

\paragraph{Detected Obstacle Rating:} The \emph{detected obstacle rating} defines UAS chances to encounter detected obstacle in avoidance grid $cell_{i,j,k}$. Final \emph{detected obstacle rating} is merged information (eq. \ref{eq:detectedObstacleRatingForCell}). The \emph{sensor field} can contain \emph{multiple} \emph{static obstacle sensors}.

\paragraph{Detected Obstacle Rate for LiDAR sensor:},Lets have only one sensor set as homogeneous two axis rotary LiDAR. For one $cell_{i,j,k}$ there exists set of passing LiDAR beams:

\begin{equation}
    lidar Rays(cell_{i,j,k})=
    \left\{
        \begin{aligned}
        \left[
            \begin{gathered}
                horizontal^\circ\in horizontal Offsets,\\
                vertical^\circ\in vertical Offsets
            \end{gathered}
        \right]\in\R^2:&\\
        horizontal^\circ\in cell_{i,j,k}.horizontal &Range,\\
        vertical^\circ\in cell_{i,j,k}.vertical &Range\\
        \end{aligned}
    \right\}
\end{equation}

\noindent The pair $[\theta,\varphi]$ is in homogeneous offset system given by product of discrete set of horizontal angles offsets $\Theta$ and discrete set of vertical angles offsets $\Phi$. The set $\mathscr{L}(c_{i,j,k})$ is finite countable and nonempty for any $c_{i,j,k}$, otherwise definition of avoidance grid $\mathscr{A}(t_i)$ needs to be changed.

The hit function $h:\mathscr{L}(c_{i,j,k})\times\R^n\to[0,\infty]$ returns a distance of single beam return for beam passing trough $[\theta,\varphi]\in\mathscr{L}(c_{i,j,k})$ angle offsets with vehicle state space $\vec{x}\in\R^n$. Then set of LiDAR hits $\mathscr{H}$ in cell $c_{i,j,k}$ at system state $\vec{x}$ is given as follow:
\begin{equation}
    lidar Hits(cell_{i,j,k})=\left\{
        \begin{aligned}
        \left[
            \begin{gathered}
                distance=lidarScan(),\\
                horizontal^\circ\in horizontal Offsets,\\
                vertical^\circ\in vertical Offsets
            \end{gathered}
        \right]\in\R^2:&\\
        distance \in cell_{i,j,k}.distance &Range,\\
        horizontal^\circ\in cell_{i,j,k}.horizontal &Range,\\
        vertical^\circ\in cell_{i,j,k}.vertical &Range\\
        \end{aligned}
    \right\}
\end{equation}

\noindent Probability of obstacle in case of LiDAR detection $P_{O_L}$ is given as ratio between landed hits and possible hits:
\begin{equation}
    obstacle^{LiDAR}_{cell_{i,j,k}}=\frac{lidar Hits(cell_{i,j,k})}{lidar Rays(cell_{i,j,k})}
\end{equation}


\noindent Probability of vision hindrance $P_{V_H}$ is given as supplement to probability of obstacle:
\begin{equation}\label{eq:probabilityOfVisibilityHindrance}
    hindrance^{LiDAR}_{cell_{i,j,k}}=1-\frac{lidar Hits(cell_{i,j,k})}{lidar Rays(cell_{i,j,k})}
\end{equation}

\paragraph{Cell density function.}
\noindent Let`s start with differential form of cell surface calculation eq. \ref{eq:finalCellSquare}. The target object have several hits in given cell Avoidance grid $\mathscr{A}$ cell $c_{i,j,k}$. Cell $c_i,j,k$ have following properties which are used in surface calculation:
\begin{enumerate}
    \item \textit{Horizontal span} ($\theta_s$, $\theta_e$) - defines range of horizontal scanner partition.
    \item \textit{Vertical span} ($\varphi_s$, $\varphi_e$) - defines range of vertical scanner partition.
\end{enumerate}

\noindent By rewriting eq. \ref{eq:finalCellSquare}. and using $\theta$ as horizontal range parameter and $\varphi$ as inverted vertical range parameter following surface integral is proposed:
\begin{multline}
    Area(cell_{i,j,k}) =\\ \int_{horizontal_{start}^\circ}^{horizontal_{end}^\circ}\int_{vertical_{start}^\circ}^{vertical_{start}^\circ} radius^2 \cos(vertical^\circ) \quad \text{d} vertical^
    \circ\text{d} horizontal^\circ
\end{multline}

\noindent Numerical stable integration exist for boundaries $\theta \ in [-\pi,\pi]$ $\varphi \in [0,\frac{pi}{2}]$ and is given by following equation:

\begin{multline}\label{eq:intersectionSurfaceForCell}
    Area(radius,horizontal Range, vertical_{start}^\circ, vertical_{end}^\circ) =\dots\\ 
    =\left\{
    \begin{aligned}
        vertical&_{start}^\circ <0, vertical_{end}^\circ \le 0 :\\ 
            &radius^2(\sin |vertical_{start}^\circ| - \sin|vertical_{end}^\circ|)\times horizontal Range\\
         vertical&_{start}^\circ <0, vertical_{end}^\circ > 0   :\\
            & r^2(\sin |vertical_{start}^\circ| + \sin|vertical_{end}^\circ|)\times horizontal Range\\
         vertical&_{start}^\circ \ge 0 vertical_{end}^\circ < 0 :\\
            & r^2(\sin vertical_{end}^\circ- \sin vertical_{start}^\circ)\times horizontal Range)
    \end{aligned}
    \right.
\end{multline}

\noindent Intersection surface for cell $A_C$ is then given by eq. \ref{eq:intersectionSurfaceForCell}. Most of the LiDAR scanners have \textit{Points per rotation parameter [ppr]}. $P_S[p/2\pi]$Let`s  assume that scanning density is homogeneous. The covered area at distance $r$ for one LiDAR swipe $A_S$ is given by eq. \ref{eq:intersectionSurfaceForCell}. where $\theta_s=\pi$, $\theta_e=\pi$, for full swipe and $\varphi_s$, $\varphi_e$ set to maximal horizontal range. Then for object with frontal area surface $A_O$, the minimal hit count $P_O$ is given as:

\begin{equation}
    lidar Hit Area(cell_{i,j,k}) = \frac{lidar Hits(cell_{i,j,k})}{lidar Rays(cell_{i,j,k})} \times Area\left(\begin{gathered}radius,horizontal Range,\\ vertical_{start}^\circ, vertical_{end}^\circ\end{gathered}\right)
\end{equation}
\begin{equation}\label{eq:lidarHitFormula}
    obstacle(LiDAR,cell_{i,j,k})=\min\left\{\frac{lidar Hit Area(cell_{i,j,k})}{UAS.lidar Threshold Area},1\right\}
\end{equation}

\noindent \textit{LiDAR hit formula} (\ref{eq:lidarHitFormula}) covers hit count for cell $c_{i,j,k}$ and object $O$ of any size $A$ and any detection distance $r$.

\section{Visibility probability}\label{sec:visibilityProbability}
\noindent For each cell $c_{i,j,k}$ and each sensor $s_k$ there exist hindrance of vision probability $P_{V_H}$, which defines how much vision is clouded in single cell. Example of hindrance calculation for LiDAR has been given by eq. \ref{eq:probabilityOfVisibilityHindrance}. Let us consider cell row $\mathscr{C}(j_{fix},k_{fix})$ with fixed horizontal index $j_fix$ and vertical index $k_{fix}$ is given as series of:

\begin{equation}\label{eq:cellrowDefinition}
    cellRow(j_{fix},k_{fix})= \left\{cell_{i,j,k}\in Avoidance Grid :\begin{aligned}&i\in\{1,..,layersCount\},\\&j=j_{fix}, k=k_{fix}\end{aligned}\right\}
\end{equation}

For each cell $c_{i,j,k}$ there exists a function which calculates final visibility hindrance probability $P_{V_F}(c_{i,j,k},\mathscr{S})$, where $\mathscr{S}$ is set of sensors providing hindrance rate. Then for ordered cell row $\mathscr{C}(j_{fix},k_{fix})=\left\{c_{1,j_{fix},k_{fix}}, c_{2,j_{fix},k_{fix}}, \dots,c_{a,j_{fix},k_{fix}}\right\}, a\in\N^+$ and for one selected cell $c_{i,j,k}$ the probability of visibility is given as supplement to hindrance from previous cells. The equation for this statement holds as follows:

\begin{multline}\label{eq:FinalVisibilityProbability}
    visibility(cell_{i_c,j_c,k_c})=\dots\\ \dots =1 - \sum_{index\in\N^+}^{index < i_c} hindrance(cell_{a,j_c,k_c}: cell_{a,j_c,k_c}\in cell Row(j_{c},k_{c})
\end{multline}

\noindent For example cell $c_{4,j_{fix},k_{fix}}$ is selected for visibility calculation $P_{V}(c_{4,j_{fix},k_{fix}})$, then cells $c_{1,j_{fix},k_{fix}}$, $c_{2,j_{fix},k_{fix}}$, and $c_{3,j_{fix},k_{fix}}$, are used as a base of calculation for visibility hindrance probability $P_{V_F}$.

\noindent \emph{The maximum hindrance} for any cell row $|{C}(j_{fix},k_{fix})|=k\in \N^+$ with  boundary:
\begin{equation}
    0 \le \sum_{cell\in cell Row(j_{fix},k_{fix})} visibility(cell) \le 1
\end{equation}

\noindent For one cell row $\mathscr{C}(j_{fix},k_{fix})$, where count of layers is equal to 10, and layers have equal spacing. There is defined sensor field $\mathscr{S}=\left\{s_1\right\}$, where $s_1$ is LiDAR sensor. 

During consequent LiDAR scans $s(t_0)$, $s(t_1)$, $s(t_2)$, and $s(t_3)$ the obstacle sets $\mathscr{O}_1(t_1)=\{o_1\}$, $\mathscr{O}_2(t_2)=\{o_1,o_2\}$, and $\mathscr{O}_3(t_3)=\{o_1,o_2,o_3\}$ are discovered. Assigned hindrance probabilities are like follow:
\begin{enumerate}
    
    \item\emph{Time $t_0$} - there is no obstacle nor hindrance, all cells are free
    \item\emph{Time $t_1$} (fig. \ref{fig:fistObstacleHindrance}) - $\mathscr{O}_1(t_1)=\{o_1\}$ was detected, the hindrance probability $P_{V_F}$ $(c_{3,j_{fix},k_{fix}})$ is equal to $0.25$. The visibility in cells $c_{4-10,j_{fix},k_{fix}}$ is 75 percent now. 
    
    \item\emph{Time $t_2$} (fig. \ref{fig:secondObstacleHindrance}) - $\mathscr{O}_2(t_2)=\{o_1,o_2\}$ was detected, the additional hindrance probability  $P_{V_F}(c_{5,j_{fix},k_{fix}})$ is $0.15$. The visibility in cells $c_{6-10,j_{fix},k_{fix}}$ is lowered by additional 15 percent and its set to 60 percent now.
    
    \item\emph{Time $t_3$} (fig. \ref{fig:thirdObstacleHindrance}) - $\mathscr{O}_3(t_3)=\{o_1,o_2,o_3\}$  was detected the additional hindrance probability  $P_{V_F}(c_{7,j_{fix},k_{fix}})$ is $0.20$. The visibility in cells $c_{8-10,j_{fix},k_{fix}}$ is lowered by additional 20 percent and its set to 40 percent now.
\end{enumerate}

\begin{figure}[H]
    \centering
    \begin{subfigure}{0.32\textwidth}
        \includegraphics[width=0.9\linewidth]{\FIGDIR/TE006VisibilityFirstObstacle} 
        \caption{1\textsuperscript{st} hindrance.}
        \label{fig:fistObstacleHindrance}
    \end{subfigure}
    \begin{subfigure}{0.32\textwidth}
        \includegraphics[width=0.9\linewidth]{\FIGDIR/TE007VisibilitySecondObstacle} 
        \caption{2\textsuperscript{nd} hindrance.}
        \label{fig:secondObstacleHindrance}
    \end{subfigure}
    \begin{subfigure}{0.32\textwidth}
        \includegraphics[width=0.9\linewidth]{\FIGDIR/TE008VisibilityThirdObstacle} 
        \caption{3\textsuperscript{rd} hindrance.}
        \label{fig:thirdObstacleHindrance}
    \end{subfigure}
    \caption{Obstacle hindrance impact on visibility in \emph{Avoidance Grid Slice}.}
    \label{fig:hindranceImpactOnVisibility}
\end{figure}


\section{Map obstacle probability}\label{sec:mapObstacleProbability}
\paragraph{Idea:} Use \emph{stored LiDAR readings} from previous mission to build an compact obstacle map \cite{cernamaria2018}. Then use \emph{this map} as a additional information source.

\paragraph{Concept:} A \emph{map obstacle} state has very simple logic, there are three possible cases:

\begin{enumerate}
    \item \emph{Undetected} - Map obstacle $O_M$ is charted on map (fig. \ref{fig:undetectedMapObstalce}), but is undetected by any sensor in sensor field, therefore the probability of map obstacle occurrence is equal to $0$.


    \item \emph{Detected} Map obstacle $O_M$ is charted on map and detected by any sensor in sensor field (fig. \ref{fig:detectedMapObstacle}). The map obstacle rate is equal to detected obstacle rate, usually its equal to $1$.

    \item \emph{Hindered} Map obstacle $O_M$ is hindered behind other detected obstacle $O_1$ (fig. \ref{fig:hinderedMapObstacle}). The detected obstacle $O_1$ is in $cell_{i,j,k}$ and is reducing visibility in follow up $cellRow_{i_f>i,j,k}$ by $60$ percent.
\end{enumerate}

\begin{figure}[H]
    \begin{subfigure}{0.32\textwidth}
        \includegraphics[width=0.9\linewidth]{\FIGDIR/TE009MapObstacleUndetected} 
        \caption{Undetected.}
        \label{fig:undetectedMapObstalce}
    \end{subfigure}
    \begin{subfigure}{0.32\textwidth}
        \includegraphics[width=0.9\linewidth]{\FIGDIR/TE010MapObstacleDetected} 
        \caption{Detected.}
        \label{fig:detectedMapObstacle}
    \end{subfigure}
    \begin{subfigure}{0.32\textwidth}
        \includegraphics[width=0.9\linewidth]{\FIGDIR/TE011MapObstacleHiden}
        \caption{Hindered.}
        \label{fig:hinderedMapObstacle}
    \end{subfigure}
    \caption{Map obstacle states after \emph{Data fusion}.}
    \label{fig:mapObstacleStatesAfterDataFusion}
\end{figure}

\paragraph{Implementation:} The formulation of final map obstacle rate  $map(cell_{i,j,k})$ was outlined in previous examples. These examples are showing the \emph{desired behaviour} and its solved by \emph{data fusion} (sec. \ref{s:sensorFusion}).

First we start with obstacle map definition. The obstacle map  (eq. \ref{eq:obstacleMap}) defines an map obstacle set of information vectors with position in global coordinate frame , orientation bounded to global coordinate reference frame, safety margin and additional parameters.
\begin{equation}\label{eq:obstacleMap}
    obstacle Map= 
    \left\{
    \begin{bmatrix}
        position,\\
        orientation,\\
        safety Margin,\\
        parameters
    \end{bmatrix}
    :
    \begin{aligned}
        & position \in  \R^3(GCF),\\
        & orientation \in \R^3(GCF),\\
        & safety Margin \in \R^+(m),\\
        & parameters \in \{\dots\}
    \end{aligned}
    \right\}
\end{equation}


The \emph{Map Obstacle} concept is taken from my \emph{master student work} \cite{cernamaria2018}, implementing \emph{compact representation} of point-cloud obstacle map. Te example of \emph{cuboid obstacles} with \emph{safe zone} is given in (fig. \ref{fig:exampleExtractedMapObstacles}).
    
\begin{figure}[H]
    \centering
    \includegraphics[width=0.7\textwidth]{\FIGDIR/TE054ExtractedMapObstaclesExample}
    \caption{Example of Extracted Map Obstacle \cite{cernamaria2018}.}
    \label{fig:exampleExtractedMapObstacles}
\end{figure} 

\noindent The space covered by any obstacle  is non-empty by definition. There are following types of map charted obstacles which are implemented in framework:

\begin{enumerate}
    \item\emph{Ball obstacle $parameters=\varnothing$} - simple ball with center at $position$, with offset safety margin.
    
    \item\emph{Line obstacle $parameters=[length]$} - simple line bounded by length $\in]0,\infty[$ with center at $position$ and given orientation with respect to main axis in global coordinate frame, with safety margin $<$ 0.
    
    \item\emph{Plane obstacle $parameters=[length,width]$} - bounded rectangle plane partition defined by length $\in]0,\infty[$, and width $w\in]0,\infty[$ with center at $\vec{p}$ and given orientation $\vec{o}$ with respect to main axis in global coordinate frame, with safety margin.
    
    \item\emph{Cuboid obstacle $parameters=[length,width,depth]$} - bounded cuboid space partition defined by length $\in]0,\infty[$, width $\in]0,\infty[$, and depth $d\in]0,\infty[$ with center at $position$ and rotated in orientation with respect to main axis in global coordinate frame, with safety margin.
\end{enumerate}

\noindent The \emph{map obstacles} are stored in clustered database. The \emph{selection criterion} is given in (eq. \ref{eq:mapObstacleSelectionCriterion}).

\begin{equation}\label{eq:mapObstacleSelectionCriterion}
    avoidance Grid.radius \ge distance(UAS.position,map Obstacle) - total Margin
\end{equation}

\noindent The \emph{total margin} is combination of \emph{safety margin} and \emph{body margin} (in case of line, plane, cuboid obstacle). The \emph{selection} was implemented as standard cluster select, selecting 26  surrounding clusters around UAS + own UAS cluster.

The \emph{compact obstacle representation} is transformed into \emph homogeneous point-cloud representations:

\begin{itemize}
    \item[1.]\emph{Body Point-cloud} - representing obstacle body approximation by geometrical shape (eq. \ref{eq:mapBodyPointCloud}). This point cloud is considered as hard constraints.
    
    \begin{equation}\label{eq:mapBodyPointCloud}
        body Point Cloud  =\{point\in\R^3(GCF): point \in map Obstacle Body\}
    \end{equation}
    
    \item[2.]\emph{Safety Margin Point Cloud} - representing safety coating around mapped obstacle body approximation (eq. \ref{eq:mapMarginPointCloud}). This point cloud is considered as soft constraint.
    \begin{equation}\label{eq:mapMarginPointCloud}
        margin Point Cloud = \{point\in\R^3(GCF): point \in map Safety Margin\}
    \end{equation}
\end{itemize}

\begin{note}
    The \emph{safety margin point cloud} is hollow in relationship to an \emph{body point cloud}, therefore:
    \begin{equation*}
        body Point Cloud \cap margin Point Cloud  = \varnothing
    \end{equation*}
\end{note}

\noindent The \emph{map obstacle} discretization to point cloud leads to problem how to calculate \emph{impact rate}. The \emph{theoretical impact rate} for \emph{obstacle} is given as:
\begin{equation*}
    impact Rate = \frac{volume(map Obstacle\cap cell_{i,j,k})}{volume(cell_{i,j,k})}\in [0,1]
\end{equation*}

\noindent The \emph{map obstacle related point clouds} (eq. \ref{eq:mapBodyPointCloud}, \ref{eq:mapMarginPointCloud}) are homogeneous \cite{cernamaria2018}. That means \emph{each point} in point clouds covers similar portion of object volume. There is \emph{threshold volume} (eq. \ref{eq:tresholdVolumeDefinition}) which represents minimal object volume to be considered as an \emph{obstacle}.

\begin{equation}\label{eq:tresholdVolumeDefinition}
    0< threshold Volume \le \frac{volume(point Cloud)}{|point Cloud|}
\end{equation}

\noindent The \emph{impact rate} of one point  when intersecting a $cell_{i,j,k}$ is given as count of \emph{threshold obstacle bodies} in \emph{point cloud covered mass} multiplied by inverted point count (eq. \ref{eq:pointImpactRateMap}).

\begin{equation}\label{eq:pointImpactRateMap}
    point. rate = \frac{point Cloud Volume}{threshold Volume}\times\frac{1}{|point Cloud|}
\end{equation}

\noindent The \emph{intersection set} between \emph{point cloud} and $cell_{i,j,k}$ is defined in (eq. \ref{eq:pointImpactRateMap}). The \emph{cell} intersection with points is defined in (eq. \ref{eq:boundedSpaceCell}).

\begin{multline}\label{eq:pointcloudIntersectionMap}
    intersection(map,cell_{i,j,k}) =\dots\\\dots \{points \in \R^3: (point\to Avoidance Grid Frame) \in cell_{i,j,k}\}
\end{multline}

\noindent The \emph{map obstacle rating} for $cell_{i,j,k}$ and obstacle for our \emph{information source} is defined in (eq. \ref{eq:mapcellratingourMap}).

\begin{equation}\label{eq:mapcellratingourMap}
    map(cell_{i,j,k},obstacle) =\max\left\{ \sum_{\forall point\in intersection(map,cell_{i,j,k})}  point.rate , 1\right\}
\end{equation}

\noindent The \emph{map obstacle rating} for $cell_{i,j,k}$ and \emph{our information source} is given as maximum of all possible cumulative ratings form each obstacle in \emph{active map obstacles} set (eq. \ref{eq:cumulativeMapCellRatingMap}).

\begin{equation}\label{eq:cumulativeMapCellRatingMap}
    map(cell_{i,j,k} = \max \left\{map(cell_{i,j,k},obstacle):\forall obstacle \in Active Map Obstacles\right\}
\end{equation}

\begin{note}
    The \emph{body point clouds} (eq. \ref{eq:mapBodyPointCloud}) never intersects, because they are created for inclusive obstacles. The \emph{safety margin point clouds} (eq. \ref{eq:mapMarginPointCloud}) can intersects, because they represents protection zones around physical obstacles. Therefore the \emph{maximum obstacle rating} (eq. \ref{eq:cumulativeMapCellRatingMap}) needs to be selected.
\end{note}

